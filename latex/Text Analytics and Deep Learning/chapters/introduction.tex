\section{Introduction}

Politics is one of the most discussed, controversial and at the same time complex fields in society.
With so many different parties advocating their positions on a wide range of issues, it's easy to lose track of what's going on. During the election campaign, many media outlets in Germany try to summarize their positions in order to inform people about current, controversial issues through interviews, news broadcasts or inviting some politicians to discuss.

Addressing this problem and making politics more accessible and less time-consuming by simplifying the views of different parties on policy issues without loss of information is a complex goal to achieve. It is difficult to inform oneself independently since several sources must be checked for each statement in order to prevent a statement being misunderstood or even faked. The complexity of this process is increased by the fact that in many instances the candidates' statements are not taken from their official websites and are often completely different from their party's own program.   Therefore, it is necessary to take a deeper look into the election programs of the parties, since it is the only source in which the parties themselves express their position and all issues are contained in a single document. 

In order to present the opinions of different parties in a more approachable way for an interested citizen, an interactive solution is quite suitable. There are some approaches such as Wahl-O-Mat\footnote{\url{https://www.wahl-o-mat.de}}, which curate questions and let the press spokespersons of the parties answer them. However, this approach has the problem that the topics and weighting are determined by the editors of Wahl-O-Mat and do not correspond to the party-political weighting. To solve this problem, our work presents an application that identifies topics and enables users to compare the different parties based purely on the party programs. By giving the public an immediate and unbiased overview of the party positions, we hope to present information that is useful to journalists, citizens and politicians.

To achieve the best possible result, we combine different state-of-the-art approaches for topic modeling in our application. 

In the following Section, we will give a brief overview of related work. Next, we present the concepts of models that are applied in our work. Finally, we present our implementation, which is broken down into five parts: preparation, procedure, results, discussion and the presentation of our web interface. 

